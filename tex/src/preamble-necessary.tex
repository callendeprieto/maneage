%% Necessary (independent of style) macros for this project.
%%
%% These are a set of packages that have been commonly necessary in most
%% LaTeX usages. However, if any are not needed in your work, please feel
%% free to remove them.
%
%% Copyright (C) 2018-2019 Mohammad Akhlaghi <mohammad@akhlaghi.org>
%
%% This template is free software: you can redistribute it and/or modify it
%% under the terms of the GNU General Public License as published by the
%% Free Software Foundation, either version 3 of the License, or (at your
%% option) any later version.
%
%% This template is distributed in the hope that it will be useful, but
%% WITHOUT ANY WARRANTY; without even the implied warranty of
%% MERCHANTABILITY or FITNESS FOR A PARTICULAR PURPOSE.  See the GNU
%% General Public License for more details.
%
%% You should have received a copy of the GNU General Public License along
%% with this template. If not, see <http://www.gnu.org/licenses/>.





%% Values from the analysis.
\input{tex/build/macros/project.tex}





% Macros for to help in typing, remove them if you don't need them, but
% this can help as a demo on how you can simply writing of commonly used
% words that need special formatting (like software names).
\newcommand{\snsign}{{\small S}/{\small N}}
\newcommand{\originsoft}{\textsf{ORIGIN}}
\newcommand{\sextractor}{\textsf{SE\-xtractor}}
\newcommand{\noisechisel}{\textsf{Noise\-Chisel}}
\newcommand{\makecatalog}{\textsf{Make\-Catalog}}





%% For highlighting updates. When this is set, text marked as \new
%% will be colored in dark green and text that is marked wtih \tonote
%% will be marked in dark red.
\ifdefined\highlightchanges
\newcommand{\new}[1]{\textcolor{green!60!black}{#1}}
\newcommand{\tonote}[1]{\textcolor{red!60!black}{[#1]}}
\else
\newcommand{\new}[1]{\textcolor{black}{#1}}
\newcommand{\tonote}[1]{{}}
\fi





% Better than verbatim for displaying typed text.
\usepackage{alltt}





% For arithmetic opertions within LaTeX
\usepackage[nomessages]{fp}





%To add a code font to the text:
\usepackage{courier}





%To add some enumerating styles
\usepackage{enumerate}





%Including images if necessary
\usepackage{graphicx}
