%% General paper's style settings.
%%
%% This preamble can be completely ignored when including this TeX file in
%% another style. This is done because this LaTeX build is meant to be an
%% initial/internal phase or part of a larger effort, so it has a basic
%% style defined here as a preamble. To ignore it, uncomment or delete the
%% respective line in `paper.tex'.



%% Print size
\usepackage[a4paper, includeheadfoot, body={18.7cm, 24.5cm}]{geometry}




%% Set the distance between the columns if two columns:
\setlength{\columnsep}{0.75cm}





% To allow figures to take up more space on the top of the page:
\renewcommand{\topfraction}{.99}
\renewcommand{\bottomfraction}{.7}
\renewcommand{\textfraction}{.05}
\renewcommand{\floatpagefraction}{.99}
\renewcommand{\dbltopfraction}{.99}
\renewcommand{\dblfloatpagefraction}{.99}
\setcounter{topnumber}{1}
\setcounter{bottomnumber}{0}
\setcounter{totalnumber}{2}
\setcounter{dbltopnumber}{1}



%% Color related settings:
\usepackage{xcolor}
\color{black}                   % Text color
\definecolor{DarkBlue}{RGB}{0,0,90}


% figure and figure* ordering correction:
\usepackage{fixltx2e}



%% For editing the caption appearence. The `setspace' package defines
%% the `stretch' variable. `abovecaptionskip' is the distance between
%% the figure and the caption.
\usepackage{setspace, caption}
\captionsetup{font=small, labelfont={color=DarkBlue,bf}, skip=1pt}
\captionsetup[figure]{font={stretch=1, small}}
\setlength{\abovecaptionskip}{3pt plus 1pt minus 1pt}



%% To make the footnotes align:
\usepackage[hang]{footmisc}
\setlength\footnotemargin{10pt}



%For including time in the title:
\usepackage{datetime}



%To make links to webpages and include document information in the
%properties of the PDF
\usepackage[
  colorlinks,
  urlcolor=blue,
  citecolor=blue,
  linkcolor=blue,
  linktocpage]{hyperref}
\renewcommand\UrlFont{\rmfamily}



% Basic Document information that goes into the PDF meta-data.
\hypersetup
{
    pdfauthor={YOUR NAME},
    pdfsubject={A SHORT DESCRIPTION OF THE WORK},
    pdftitle={THE TITLE OF THIS PROJECT},
    pdfkeywords={SOME, KEYWORDS, FOR, THE, PDF}
}



% Title, author, pipeline info and date as they appear on the output PDF.
\title{THE TITLE OF THIS PROJECT}
\author{YOUR NAME, COLLEAGE1 NAME, ETC}
\date{\small Reproduction pipeline \pipelineversion{}
      and Gnuastro \gnuastroversion\\on \today, \currenttime}
