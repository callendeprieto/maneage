\documentclass[10pt, twocolumn]{article}

%% This is a convenience variable if you are using PGFPlots to build plots
%% within LaTeX. If you want to import PDF files for figures directly, you
%% can use the standard `\includegraphics' command. See the definition of
%% `\includetikz' in `tex/preamble-pgfplots.tex' for where the files are
%% assumed to be if you use `\includetikz' when `\makepdf' is not defined.
\newcommand{\makepdf}{}

%% When defined (value is irrelevant), `\highlightchanges' will cause text
%% in `\tonote' and `\new' to become colored. This is useful in cases that
%% you need to distribute drafts that is undergoing revision and you want
%% to hightlight to your colleagues which parts are new and which parts are
%% only for discussion.
\newcommand{\highlightchanges}{}

%% Necessary LaTeX preambles to include for relevant functionality. We want
%% to start this file as fast as possible with the actual body of the
%% paper, while keeping modularity in the preambles.
\input{tex/pipeline.tex}
%% General paper's style settings.
%%
%% This preamble can be completely ignored when including this TeX file in
%% another style. This is done because this LaTeX build is meant to be an
%% initial/internal phase or part of a larger effort, so it has a basic
%% style defined here as a preamble. To ignore it, uncomment or delete the
%% respective line in `paper.tex'.





%% Print size
\usepackage[a4paper, includeheadfoot, body={18.7cm, 24.5cm}]{geometry}





%% Set the distance between the columns if two columns:
\setlength{\columnsep}{0.75cm}





% To allow figures to take up more space on the top of the page:
\renewcommand{\topfraction}{.99}
\renewcommand{\bottomfraction}{.7}
\renewcommand{\textfraction}{.05}
\renewcommand{\floatpagefraction}{.99}
\renewcommand{\dbltopfraction}{.99}
\renewcommand{\dblfloatpagefraction}{.99}
\setcounter{topnumber}{1}
\setcounter{bottomnumber}{0}
\setcounter{totalnumber}{2}
\setcounter{dbltopnumber}{1}





%% Color related settings:
\usepackage{xcolor}
\color{black}                   % Text color
\definecolor{DarkBlue}{RGB}{0,0,90}






% figure and figure* ordering correction:
\usepackage{fixltx2e}





%% For editing the caption appearence. The `setspace' package defines
%% the `stretch' variable. `abovecaptionskip' is the distance between
%% the figure and the caption.
\usepackage{setspace, caption}
\captionsetup{font=footnotesize, labelfont={color=DarkBlue,bf}, skip=1pt}
\captionsetup[figure]{font={stretch=1, small}}
\setlength{\abovecaptionskip}{3pt plus 1pt minus 1pt}
\setlength{\belowcaptionskip}{-1.25em}






%% To make the footnotes align:
\usepackage[hang]{footmisc}
\setlength\footnotemargin{10pt}





%For including time in the title:
\usepackage{datetime}





%To make links to webpages and include document information in the
%properties of the PDF
\usepackage[
  colorlinks,
  urlcolor=blue,
  citecolor=blue,
  linkcolor=blue,
  linktocpage]{hyperref}
\renewcommand\UrlFont{\rmfamily}





%% Define the abstract environment
\renewenvironment{abstract}
 {\vspace{-0.5cm}\small%
  \list{}{%
    \setlength{\leftmargin}{2cm}%
    \setlength{\rightmargin}{\leftmargin}%
  }%
  \item\relax}
 {\endlist}





%% To keep the main page's code clean.
\newcommand{\includeabstract}[1]{%
\twocolumn[%
  \begin{@twocolumnfalse}%
    \maketitle%
    \begin{abstract}%
    #1%
    \end{abstract}%
    \vspace{1cm}%
  \end{@twocolumnfalse}%
  ]%
}





% Basic Document information that goes into the PDF meta-data.
\hypersetup
{
    pdfauthor={YOUR NAME},
    pdfsubject={A SHORT DESCRIPTION OF THE WORK},
    pdftitle={THE TITLE OF THIS PROJECT},
    pdfkeywords={SOME, KEYWORDS, FOR, THE, PDF}
}





% Title, author, pipeline info and date as they appear on the output PDF.
\title{\vspace{-3em}THE TITLE OF YOUR PROJECT}
\author{YOUR NAME, COLLEAGE1 NAME, ETC}
\date{\small Reproduction pipeline \pipelineversion{}
      and Gnuastro \gnuastroversion\\on \today, \currenttime}

%% The headers: title, authors, top of pages and section title formatting
%% of the final LaTeX file are configured here.





%% General page header settings.
\usepackage{fancyhdr}
\pagestyle{fancy}
\lhead{\footnotesize{\scshape Draft paper}, {\footnotesize nnn:i (pp), Year Month day}}
\rhead{\scshape\footnotesize YOUR-NAME et al.}
\cfoot{\thepage}
\setlength{\voffset}{0.75cm}
\setlength{\headsep}{0.2cm}
\setlength{\footskip}{0.75cm}
\renewcommand{\headrulewidth}{0pt}





%% Specific style for first page.
\fancypagestyle{firststyle}
{
  \lhead{\footnotesize{\scshape Draft paper}, nnn:i (pp), YYYY Month day\\
  \scriptsize \textcopyright YYYY, Your name. All rights reserved.}
  \rhead{\footnotesize \footnotesize \today, \currenttime\\}
}





%To set the style of the titles:
\usepackage{titlesec}
\titleformat{\section}
  {\centering\normalfont\uppercase}
  {\thesection.}
  {0em}
  { }
\titleformat{\subsection}
  {\centering\normalsize\slshape}
  {\thesubsection.}
  {0em}
  { }
\titleformat{\subsubsection}
  {\centering\small\slshape}
  {\thesubsubsection.}
  {0em}
  { }





% Basic Document information that goes into the PDF meta-data.
\hypersetup
{
    pdfauthor={YOUR NAME},
    pdfsubject={A SHORT DESCRIPTION OF THE WORK},
    pdftitle={THE TITLE OF THIS PROJECT},
    pdfkeywords={SOME, KEYWORDS, FOR, THE, PDF}
}





%% Title and author information
\usepackage{authblk}
\renewcommand\Authfont{\small\scshape}
\renewcommand\Affilfont{\footnotesize\normalfont}
\setlength{\affilsep}{0.2cm}

\title{\large \uppercase{Put the title of your exciting project here}}

\author[1]{Your name}
\author[2]{Coauthor one}
\author[1,3]{Coauthor two}

\affil[1]{The first affiliation in the list.; \url{your@email.address}}
\affil[2]{Another affilation can be put here.}
\affil[3]{And generally as many affiliations as you like.
\par \emph{Received YYYY MM DD; accepted YYYY MM DD; published YYYY MM DD}}
\date{}

%% Biblatex settings.
%%
%% Since the preamble settings necessary to make the bibliography with
%% Biblatex is a little long and unclean, and might be used in other places
%% separately later, it is easier to have it here as a separate file.
%%
%% USAGE:
%%
%%  - `tex/ref.tex': the file containing Bibtex source of each
%%     reference. The file suffix doesn't have to be `.bib', this naming
%%     helps in clearly identifying the files and avoiding places that
%%     complain about `.bib' files.





%% To break up highlighted text (for example texttt when some it is on the
%% line break) and also to no underline emphasized words (like journal
%% titles in the references).
\usepackage[normalem]{ulem}





% Basic BibLaTeX settings
\usepackage[
    doi=false,
    url=false,
    dashed=false,
    eprint=false,
    maxbibnames=4,
    minbibnames=1,
    hyperref=true,
    maxcitenames=2,
    mincitenames=1,
    style=authoryear,
    uniquelist=false,
    backend=biber,natbib]{biblatex}
\DeclareFieldFormat[article]{pages}{#1}
\DeclareFieldFormat{pages}{\mkfirstpage[{\mkpageprefix[bookpagination]}]{#1}}
\addbibresource{tex/references.tex}
\renewbibmacro{in:}{}
\renewcommand*{\bibfont}{\footnotesize}
\DefineBibliographyStrings{english}{references = {References}}

%% Include the adsurl field key into those that are recognized:
\DeclareSourcemap{
  \maps[datatype=bibtex]{
    \map{
      \step[fieldsource=adsurl,fieldtarget=iswc]
      \step[fieldsource=gbkurl,fieldtarget=iswc]
    }
  }
}

%% Set the color of the doi link to mymg (magenta) and the ads links
%% to mypurp (or purple):
\definecolor{mypurp}{cmyk}{0.75,1,0,0}
\newcommand{\doihref}[2]{\href{#1}{\color{magenta}{#2}}}
\newcommand{\adshref}[2]{\href{#1}{\color{mypurp}{#2}}}
\newcommand{\blackhref}[2]{\href{#1}{\color{black}{#2}}}

%% Define a format for the printtext commands in
%% DeclareBibliographyDriver to make links for the doi, ads link and
%% arxiv link:
\DeclareFieldFormat{doilink}{
  \iffieldundef{doi}{#1}{\doihref{http://dx.doi.org/\thefield{doi}}{#1}}}
\DeclareFieldFormat{adslink}{
    \iffieldundef{iswc}{#1}{\adshref{\thefield{iswc}}{#1}}}
\DeclareFieldFormat{arxivlink}{
  \iffieldundef{eprint}{#1}{\href{http://arxiv.org/abs/\thefield{eprint}}{#1}}}

\DeclareListFormat{doiforbook}{
  \iffieldundef{doi}{#1}{\doihref{http://dx.doi.org/\thefield{doi}}{#1}}}
\DeclareFieldFormat{googlebookslink}{
    \iffieldundef{iswc}{#1}{\adshref{\thefield{iswc}}{#1}}}

%% Set the formatting to make the last three values into the
%% appropriate link. Note that the % signs are necessary. Without
%% them, the items will be indented.
\DeclareBibliographyDriver{article}{%
  \usebibmacro{bibindex}%
  \usebibmacro{begentry}%
  \usebibmacro{author/translator+others}%
  \newunit%
  \ifdefined\makethesis\printtext{\usebibmacro{title}}\fi%
  \newunit%
  \printtext[doilink]{\usebibmacro{journal}}%
  \addcomma%
  \printtext[adslink]{\printfield{volume}}%
  \addcomma%
  \printtext[arxivlink]{\printfield{pages}}%
  \addperiod%
}

\DeclareBibliographyDriver{book}{%
  \usebibmacro{bibindex}%
  \usebibmacro{begentry}%
  \usebibmacro{author/translator+others}%
  \newunit%
  \printtext{\usebibmacro{title}}%
  \addperiod%
  \addspace%
  \printlist[doiforbook]{publisher}%
  \addcomma%
  \addspace%
  \printfield[googlebookslink]{edition}%
  \printtext{ ed.}%
  \addperiod%
}

%% In order to have et al. instead of et al.,:
\renewcommand*{\nameyeardelim}{\addspace}

%% PGFPlots settings
%% -----------------
%%
%% PGFPLOTS is a package in (La)TeX for making plots internally. It fits
%% nicely with the purpose of a reproduction pipeline. But it isn't
%% mandatory. Therefore if you don't need it, just comment/delete the line
%% that includes this file in the top LaTeX source (`paper.tex').
%%
%% However, TiKZ and PGFPlots are the recommended way to include figures
%% and plots in your paper. There are two main reasons: 1) it follows the
%% same LaTeX settings as the text of the paper, so the figures will be in
%% the exact same settings (for example font or lines) as the main body of
%% the papers. 2) It doesn't require any extra dependency (it is
%% distributed as part of TeX-live). Adding specific programs/libraries for
%% plots can greatly increase the number of dependencies for the
%% pipeline. For example Python's Matplotlib library is indeed very good,
%% but it requires Python and Numpy. The latter is not easy to build from
%% source, so after a few years, installing the required version can be
%% very frustrating.
%%
%% Keeping all BibLaTeX settings in a separate preamble was done in the
%% spirit of modularity to 1) easily managable, 2) If a similar BibLaTeX
%% configuration is necessary in another LaTeX compilation, this file can
%% just be copied there and used.
%%
%% PGFPlots uses the (La)TeX TiKZ package to build plots. So we will first
%% do the settings that are necessary in TiKZ, and then go onto the actual
%% PGFPlots package.
%%
%% USAGE:
%%
%%  - All plots are made within a `tikz' directory (that must already be
%%    present in the location LaTeX is run).
%%
%%  - Use `\includetikz{XXXX}' to make/use the figure. If a `makepdf' LaTeX
%%    macro is not defined, then it will simply assume a `XXXX.pdf' file
%%    exists in the `\bdir/tex/build/tikz' directory and simply import
%%    it. If `makepdf' is defined, then TiKZ/PGFPlot will be called to
%%    (possibly) build the plot based on `tex/XXXX.tex'. Note that if the
%%    contents of `tex/XXXX.tex' hasn't changed since the las
%%    build. TiKZ/PGFPlots won't rebuild the plot.





%% Very general TiKZ settings. In particular, to allow faster processing
%% (not having to re-build the plots on every run), we are using the
%% externalization feature of TiKZ. With this option, TiKZ will build every
%% figure independently in a special directory afterwards it will include
%% the built figure in the final file. This has many advantages: 1) if the
%% code for the plot hasn't changed, then the plot won't be re-made (can be
%% slow with detailed plots). 2) You can use the PDFs of the individual
%% plots for other purposes (for example to include in slides) cleanly.
\usepackage{tikz}
\usetikzlibrary{external}
\tikzexternalize
\tikzsetexternalprefix{tikz/}





%% The following rule will cause the name of the files keeping a figure's
%% external PDF to be set based on the file that the TiKZ commands are
%% from. Without this, TiKZ will use numbers based on the order of
%% figures. These numbers can be hard to manage and they will also depend
%% on order in the final PDF, so it will be very buggy to manage them.
\newcommand{\includetikz}[1]{%
  \ifdefined\makepdf%
    \tikzsetnextfilename{#1}%
    \input{tex/#1.tex}%
  \else
    \includegraphics[width=\linewidth]{\bdir/tex/build/tikz/#1.pdf}
  \fi
}





%% Uncomment the following lines for EPS and PS images. Note that you still
%% have to use the `pdflatex' executable and also add a `[dvips]' option to
%% graphicx.

%% \tikzset{external/system call={rm -f "\image".eps "\image".ps
%% "\image".dvi; latex \tikzexternalcheckshellescape -halt-on-error
%% -interaction=batchmode -jobname "\image" "\texsource";
%% dvips -o "\image".ps "\image".dvi;
%% ps2eps "\image.ps"}}





%% Inport and configure PGFPlots.
\usepackage{pgfplots}
\pgfplotsset{compat=newest}
\usepgfplotslibrary{groupplots}
\pgfplotsset{
  axis line style={thick},
  tick style={semithick},
  tick label style = {font=\footnotesize},
  every axis label = {font=\footnotesize},
  legend style = {font=\footnotesize},
  label style = {font=\footnotesize}
  }

%% Some (commonly) necessary LaTeX packages.
%%
%% These are a set of packages that are commonly necessary in most LaTeX
%% usages. However, if any are not needed in your work, you can remove them
%% from here.





% Macros for to help in typing, remove them if you don't need them, but
% this can help as a demo on how you can simply writing of commonly used
% words that need special formatting (like software names).
\newcommand{\snsign}{{\small S}/{\small N}}
\newcommand{\originsoft}{\textsf{ORIGIN}}
\newcommand{\sextractor}{\textsf{SE\-xtractor}}
\newcommand{\noisechisel}{\textsf{Noise\-Chisel}}
\newcommand{\makecatalog}{\textsf{Make\-Catalog}}





%% For highlighting updates. When this is set, text marked as \new
%% will be colored in dark green and text that is marked wtih \tonote
%% will be marked in dark red.
\ifdefined\highlightchanges
\newcommand{\new}[1]{\textcolor{green!60!black}{#1}}
\newcommand{\tonote}[1]{\textcolor{red!60!black}{[#1]}}
\else
\newcommand{\new}[1]{\textcolor{black}{#1}}
\newcommand{\tonote}[1]{{}}
\fi





% Better than verbatim for displaying typed text.
\usepackage{alltt}





% For arithmetic opertions within LaTeX
\usepackage[nomessages]{fp}





%To add a code font to the text:
\usepackage{courier}





%To add some enumerating styles
\usepackage{enumerate}





%Including images if necessary
\usepackage{graphicx}











%% Start writing.
\begin{document}

%% Abstract, keywords and reproduction pipeline notice.
\includeabstract{

  You have completed the reproduction pipeline and are ready to configure
  and implement it for your own research. This template reproduction
  pipeline and document contains almost all the elements that you will need
  in a research project containing the downloading of raw data, processing
  it, including them in plots and report, including this abstract, figures
  and bibliography. If you use this pipeline in your work, don't forget to
  add a notice to clearly let the readers know that your work is
  reproducible. If this pipeline proves useful in your research, please
  cite \citet{ai15}.

  \vspace{0.25cm}

  \textsl{Keywords}: Add some keywords for your research here.

  \textsl{Reproducible paper}: Generated from reproduction pipeline
  \pipelineversion{} and Gnuastro \gnuastroversion. Pipeline available at:
  \url{https://link-to.the/git/repo-of-your-pipeline}.
}

%% To add the first page's headers.
\thispagestyle{firststyle}





%% Start of the main body of text.
\section{Congratulations!}
Congratulations on running the reproduction pipeline! You can now follow
the checklist in the \texttt{README.md} file to customize this pipeline to
your exciting research project.

Just don't forget to \emph{never} use any numbers or fixed strings (for
example database urls like \url{\websurvey}) directly within your \LaTeX{}
source. Read them directly from your configuration files or outputs of the
programs as part of the reproduction pipeline and import them into \LaTeX{}
as macros through the \texttt{tex/pipeline.tex} file. See the several
examples within the pipeline for a demonstration. For some recent
real-world examples, the reproduction pipelines for Sections 4 and 7.3 of
\citet{bacon17} are available at
\href{https://doi.org/10.5281/zenodo.1164774}{zenodo.1164774}\footnote{\url{https://gitlab.com/makhlaghi/muse-udf-origin-only-hst-magnitudes}},
or
\href{https://doi.org/10.5281/zenodo.1163746}{zenodo.1163746}\footnote{\url{https://gitlab.com/makhlaghi/muse-udf-photometry-astrometry}}. Working
in this way, will let you focus clearly on your science and not have to
worry about fixing this or that number/name in the text.

Just as a demonstration of creating plots within \LaTeX{} (using the
{\small PGFP}lots package), in Figure \ref{deleteme} we show a simple
plot, where the Y axis is the square of the X axis. The minimum value
in this distribution is $\deletememin$, and $\deletememax$ is the
maximum. Take a look into the \LaTeX{} source and you'll see these
numbers are actually macros that were calculated from the same dataset
(they will change if the dataset, or function that produced it,
changes).

The individual {\small PDF} file of Figure \ref{deleteme} is available
under the \texttt{tex/build/tikz/} directory of your build directory. You
can use this PDF file in other contexts (for example in slides showing your
progress or after publishing the work). If you want to directly use the
{\small PDF} file in the figure without having to let {\small T}i{\small
  KZ} decide if it should be remade or not, you can also comment the
\texttt{makepdf} macro at the top of this \LaTeX{} source file.

{\small PGFP}lots is a great tool to build the plots within \LaTeX{} and
removes the necessity to add further dependencies (to create the plots) to
your reproduction pipeline. High-level language libraries like Matplotlib
do exist to also generate plots. However, bare in mind that they require
many dependencies (Python, Numpy and etc). Installing these dependencies
from source (after several years when the binaries are no longer available
in common repositories), is not easy and will harm the reproducibility of
your paper.

\begin{figure}[t]
  \includetikz{delete-me}

  \captionof{figure}{\label{deleteme} A very basic $X^2$ plot for
    demonstration.}
\end{figure}

Furthermore, since {\small PGFP}lots is built by \LaTeX{} it respects all
the properties of your text (for example line width and fonts and etc), so
the final plot blends in your paper much more nicely. It also has a
wonderful
manual\footnote{\url{http://mirrors.ctan.org/graphics/pgf/contrib/pgfplots/doc/pgfplots.pdf}}.

This pipeline also defines two \LaTeX{} macros that allow you to mark text
within your document as \emph{new} and \emph{notes}. For example, \new{this
  text has been marked as \texttt{new}.} \tonote{While this one is marked
  as \texttt{tonote}.} If you comment the line (by adding a `\texttt{\%}'
at the start of the line or simply deleting the line) that defines
\texttt{highlightchanges}, then the one that was marked \texttt{new} will
become black (totally blend in with the rest of the text) and the one
marked \texttt{tonote} will not be in the final PDF. You can thus use
\texttt{highlightchanges} to easily make copies of your research for
existing coauthors (who are just interested in the new parts or notes) and
new co-authors (who don't want to be distracted by these issues in their
first time reading).





\section{Notice and citations}
To encourage other scientists to publish similarly reproducible papers,
please add a notice close to the start of your paper or in the end of the
abstract clearly mentioning that your work is fully reproducible.

For the time being, we haven't written a specific paper only for this
reproduction pipeline, so until then, we would be grateful if you could
cite the first paper that used the first version of this pipeline:
\citet{ai15}.

After publication, don't forget to upload all the necessary data, software
source code and the reproduction pipeline to a long-lasting host like
Zenodo (\url{https://zenodo.org}).





\section{Acknowledgements}
\new{Please include the following two paragraphs in the Acknowledgement
  section of your paper. This reproduction pipeline was developed in
  parallel with Gnuastro, so it benefited from the same grants. If you
  don't use any of these packages in the final/customized pipeline, please
  remove them. }

This research was partly done using GNU Astronomy Utilities (Gnuastro,
ascl.net/1801.009) version \gnuastroversion, and reproduction pipeline
\pipelineversion. Work on Gnuastro and the reproduction pipeline has been
funded by the Japanese Ministry of Education, Culture, Sports, Science, and
Technology (MEXT) scholarship and its Grant-in-Aid for Scientific Research
(21244012, 24253003), the European Research Council (ERC) advanced grant
339659-MUSICOS, European Union’s Horizon 2020 research and innovation
programme under Marie Sklodowska-Curie grant agreement No 721463 to the
SUNDIAL ITN, and from the Spanish Ministry of Economy and Competitiveness
(MINECO) under grant number AYA2016-76219-P.

The following free software tools were also critical component of this
research (in alphabetical order): Bzip2 \bziptwoversion, {\small CFITSIO}
\cfitsioversion, CMake \cmakeversion, c{\small URL} \curlversion, Git
\gitversion, \gnu{Bash} \bashversion, \gnu{Binutils} \binutilsversion,
\gnu{Coreutils} \coreutilsversion, \gnu{{\small AWK}} \gawkversion,
\gnu{Grep} \grepversion, \gnu{Libtool} \libtoolversion, \gnu{Make}
\makeversion, \gnu{Sed} \sedversion, \gnu{Scientific Library} ({\small
  GSL}) \gslversion, \gnu{Tar} \tarversion, Lzip \lzipversion, {\small GPL}
Ghostscript \ghostscriptversion, Libgit2 \libgitwoversion, Libtiff
\libtiffversion, {{\small WCSLIB}} \wcslibversion, {\small XZ} Utils
\xzversion, and ZLib \zlibversion. We are very grateful to all their
creators for freely providing this necessary infrastructure. This research
would not be possible without them.

%% Tell BibLaTeX to put the bibliography list here.
\printbibliography

%% Finish LaTeX
\end{document}
